


\section{Formatting and Structure}
\newthought{Use the formatting and style checklist} to confirm that your document adheres to the requirements. Feel free to further detail out the checklist to
meet your needs, according to the guidelines provided.
\marginnote{
\newthought{How to thesis} guideline is \href{pages.schutera.com/HowToThesis/notes.pdf}{here}.
}

\begin{itemize}[label={}]
	\item [$\Box$]Thesis follows required structure
	\item [$\Box$]Thesis follows required format and style guidelines
	\item [$\Box$]Citations and references are complete
	\item [$\Box$]Language and writing is clear, academic, and free of errors
	\item [$\CheckedBox$] Checking this item, so you can see how it's done.
\end{itemize}


\vfill
\section{Methodology}
\newthought{Use the methodology and reproducibility checklist} to validate that your methods, algorithms, datasets, and evaluation procedures are described with enough clarity for another researcher to understand and replicate your work. Confirm that all diagrams, schematics, and workflow figures are both present and accurate, and that all code, parameters, and tools are properly documented or referenced.

\begin{itemize}[label={}]
	\item [$\Box$] Methods and algorithms are described in detail
	\item [$\Box$]Datasets are clearly documented and accessible
	\item [$\Box$]Evaluation procedures are reproducible and transparent
	\item [$\Box$]Code, parameters, and tools are documented and deployable
	\item [$\Box$]Limitations and assumptions are stated
\end{itemize}


% You feel something fundamental is missing here? This is your moment to contribute.
% Contributions will be appreciated by everyone who comes after you - make their day.


\marginnote{
\newthought{Treat the checklists as an opportunity for quality control, not as an afterthought.} 
It is expected that you make any necessary revisions identified while completing them. 
If an item is unclear or not applicable, briefly justify in this document.
}




