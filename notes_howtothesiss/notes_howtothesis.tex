%%%%%%%%%%%%%%%%%%%%%%%%%%%%%%%%%%%%%%%%%%%%%%%%%%%%%%%%%%%%%%%%%%%%%%
% How to Write a Scientific Thesis
% Author: Dr.-Ing. Mark Schutera
% Based on Tufte-style layout
\makeindex
%%%%%%%%%%%%%%%%%%%%%%%%%%%%%%%%%%%%%%%%%%%%%%%%%%%%%%%%%%%%%%%%%%%%%%

\documentclass{../sharedAssets/tufte}

%%
% Book metadata
\title{Druidic Knowledge on Writing a Thesis}
\author[\defaultauthor]{\defaultauthor}
\publisher[\defaultpublisher]{\defaultpublisher}

% Load optional fonts
\loadoptionalfonts

% Additional packages
\usepackage{pifont}
\usepackage{amsmath}
\usepackage{soul}
\usepackage{textcase}
\usepackage{enumitem}
\usepackage{tikz}
\usepackage{pgfplots}
\usepackage{float}
\usepackage{algorithm}
\usepackage{algpseudocode}

\begin{document}



\makestandardfrontmatter
% Dedication
% Alternative quotes:
\makededication{\openepigraph{“Research is formalized curiosity. It is poking and prying with a purpose”}{Wernher von Braun}}
\tableofcontents


% Introduction
\makeintroduction{%
  We have all been there—facing that empty stack of papers or pixels for that matter, ultimately facing the daunting task of writing a scientific thesis.
  \newthought{This guide} distills years of experience into practical advice for crafting a clear, coherent, and compelling thesis.
  It covers essential aspects such as structure, writing techniques, formatting, and common pitfalls to avoid.
  
  \newthought{Feedback is invaluable} in this process. If nothing is expected, Feedback is. 
  If you have suggestions, corrections, or additional tips, please reach out. 
  Your contributions will help enhance this guide for future readers. 

  \newthought{This document is a living work and should not be considered complete}—there are approaches I have tested and refined, 
  others I am still exploring, some I will adapt based on new insights, and perspectives I thought I understood but continue to evolve.

  \newthought{Make sure to use} the knowledge distilled in this resource. 
  }

%% Start the main matter
\mainmatter


% ----------------------------------------------------------------------

\chapter{Preamble}\index{Preamble}


\section{Best Practices}\index{Best Practices}

\newthought{Select a topic}\index{Topic selection} that aligns with your interests and career goals.
Ensure the topic is feasible within the given timeframe and resources. To discuss a potential topic with a supervisor, 
prepare a brief proposal outlining your research question, objectives, and methodology. 

\newthought{Manage your responsibilities}\index{Industry cooperation} when engaging in a cooperation between industry and academia.
Students must manage both tasks carefully, as this is a great way to learn what is of interest to the practitioner, 
the requirements for the thesis and the external position may differ at times.

\newthought{Ensure the supervisor's}\index{Supervisor} research interests and expectations align with your thesis work.
  Look for supervisors with a strong track record in guiding successful theses. 
  Reflect on whether you prefer a hands-on or hands-off mentorship style and choose accordingly.

\newthought{Templates}\index{Templates}, there are usually always templates provided by your institution, or Professor, make use of them.

\newthought{Read this Guideline}\index{Guideline}, really this will take you a long way. A nice side-effect is that feedback and discussion 
rounds will be more focused on content than formatting or structure - your work and you can only benefit from that.

\newthought{Start early}\index{Start early} and divide your work into manageable efforts, with clear artifacts. Do the same for your writing - for same page you will need a whole day. 
Start early, get a first prototype or draft down, and iterate. The first thing you want to have is an outline of your thesis structure.

\marginnote{%
  \textbf{A great first outline} \\
 can be your table of contents.}

 Then go for your full introduction part, which helps you clarify your problem statements and targeted contributions. 

\newthought{Write in English}\index{English} at all cost. This is research, you are writing to be read and to make an impact with your contribution.
Don't handicap yourself by writing in a language that is not widely understood in the scientific community. Make use of our lingua franca - English (US). 


\newthought{Read your Thesis}\index{Read thesis}
You should have read your own thesis multiple times before submitting it. 
You should have read it for flow, for structure, for content, for formatting, for appearance, for typos, for grammar, for punctuation - you name it.
\marginnote{%
  \newthought{Checklists} are your friend.\\
}
You should have someone else read it as well, preferably someone who is not familiar with your work.

\newthought{Use LaTex}\index{LaTeX} to write your thesis. Personally, I can recommend Visual Studio Code combined with the extensions LaTeX Workshop (for TeX support)
and LTeX (for spell and style checking). 

\newthought{Establish a plan}\index{Planning}, it is crucial to establish a personal schedule and workflow. 
Unlike regular coursework, writing a thesis comes with no fixed timetable.
 You must decide for yourself when and how long to work, and set your own milestones. 
 Do not underestimate the importance of planning and executing the plan. 
 Develop a routine that fits your productivity peaks and other commitments. 

\newthought{Stay organized}\index{Organization}, keep track of your sources, notes, and drafts. Your thesis document, can help a lot in this, 
but should not be the only place where you structure, plan and document your experiments and work.

\newthought{Back Up your Work}\index{Backup}. Use version control to be able to recover your work. Back up regularly, better automatically.
And this is not only about your writing, but also your code, data, and any other digital artifacts you create during your research.

\newthought{Constant dripping}\index{Persistence} wears away the stone. Constant progress, even if small, accumulates over time and leads to significant results.
Work on your thesis every day, and avoid long breaks. Thoughts need time to mature and incubate in your brain, 
even when you are not actively working on your thesis.

\newthought{Motivation fluctuates.}\index{Motivation} This is normal. Progress is not linear, so frustration and self-doubt are part of the game. Embrace it, this is not only you.
In such times it is even more important to stick to the routine, divide work into small manageable tasks, and celebrate small wins. Welcome to the grind. 
\marginnote{%
  \newthought{When nothing else works}, fall back to what I call \textit{productive procrastination}—organize notes, format your writing, polish figures.
}

\newthought{When you get stuck}\index{Problem solving} there are a few strategies:
\begin{itemize}
  \item Solve an easier version of the problem first.
  \item Use another approach or method.
  \item Talk to peers, supervisors, or mentors for fresh perspectives.
\end{itemize}


\section{Brief Note on the Use of AI}\index{AI}
\textit{"The use of AI does not change the fundamental principles of academic integrity.
AI tools should be regarded as supplementary resources, similar to calculators,
and used with responsibility and understanding."} \cite{FreiburgThesisGuideline2025}.


\marginnote{
  In short. When you try to hit a nail with a hammer, you can not blame the hammer if you end up hitting your thumb.
  It will have been \textbf{you}. This should not stop you from using the hammer in the first place.\\

  Just ask yourself: \textbf{Am I producing original and thoughtful work?} You should be able to answer with yes.
}
\newthought{AI} is here to stay, it is to be understood as a tool, a tool you need to gain proficiency with.
As such I encourage you to explore the use of AI tools in your thesis and research work while being mindful of their limitations.
Do not rob yourself of valuable learning experiences. 
In the following we will outline some guidelines and guardrails to follow.

\begin{marginfigure}
  \includegraphics[width=\linewidth]{figures/math-teachers-protest-against-calculator-use.png}
  \caption{Math teachers protest against calculator use in 1966.}
\end{marginfigure}


\newthought{Whitelisted AI} allows for the use of AI without further documentation requirements, 
as proposed in~\cite{KielUniversity2025}.

\begin{itemize}
  \item Spell and grammar checking
  \item Thesaurus tools
  \item Translation assistance (excluding direct quotations)
  \item Simple proofreading (reviewing structure and logic, suggesting reformulations)
  \item Error checking in code
  \item Support in the general understanding of models, graphs, tables and/or illustrations
\end{itemize}

\newthought{AI usage beyond} those whitelisted need examiner approval. Their use must be made transparent, and follow basic principles of adacemic integrity.

\newthought{AI can be cited as such,}
\textit{"Sure feel free to cite me (ChatGPT [Large Language Model], 2025)."}









%----------------------------------------------------------------------

\chapter{Structure and Scope}\index{Structure}

\marginnote{
 \textbf{Common Thesis Structure:} \\
 1. Title Page \\
 2. Abstract \\
 3. Affirmation \\
 4. Acknowledgements \\
 5. Table of Contents \\
 6. List of Figures and Tables \\
 7. List of Equations \\
 8. List of Abbreviations \\
 9. List of Symbols \\
 10. IMRD Body \\
 11. Index \\
 12. References \\
 13. Appendices \\
 }

\newthought{The structure is for the reader.} You read that right, the structure is not primarily for you. 
Of course a good structure helps you to organize and document your thoughts and work as you go - but it would be no good advice to limit yourself to that.
Ultimately then, a well-organized thesis guides the reader through your research journey, 
helping them understand your objectives, methods, findings, contributions and conclusions.\\

\newthought{While some flexibility} is allowed and even encouraged to adapt the structure to your specific research topic and methodology,
the following sections outline a commonly accepted structure for scientific theses. The IMRD (Introduction, Methods, Results, Discussion) is a good starting point.


\section{Abstract}\index{Abstract}
\newthought{The abstract} provides a concise summary of your thesis, including the research question, 
chosen approach, key findings, and their interpretation. 
It should be no longer than half a page. While optional, feel free to write the abstract in English 
and the native language of your institution.
The abstract is best written after the main sections are complete, 
ensuring it accurately reflects the content and contributions of your work.



\section{Introduction}\index{Introduction}
\newthought{The introduction} is $20 - 30\%$ of the total thesis length and is structured into: 
\begin{itemize}
  \item Context and Motivation
  \item Related Work and Background through Literature Review
  \item Research Gap and Problem Statement
  \item Objectives and their Significance
  \item Thesis Structure Overview
\end{itemize}

\marginnote{\textbf{Lengthy motivation sections}\\
  covering \textit{how autonomous driving brings down traffic accidents by 90\%}, or reiteration of broad concepts, 
  or (humble bragging around) how awesome your industrial partner is, or similar topics, are to be avoided - A short single sentence will do.
  Then focus on motivating your specific field and research problem.
}

Begin with a broad yet short overview to provide context and motivation in the current field.

\newthought{Related Work -} or State of the Art, summarizes and connects relevant literature to highlight and organize existing knowledge and identify \textbf{gaps}. 
Sometimes this is extended to a section on preliminary, or foundational concepts - however when you go that path, don't put too much into it. 
Make use of Google Scholar as a tool to do so.
\marginnote{%
  \textbf{Gaps can be diverse}, \\
    \textit{Inconsistencies or contradictions} in existing studies. \\
    Methodological \textit{limitations} in previous research\\
    Missing \textit{transfer} of methods between fields of application. 
}
Select your sources carefully and evaluate whether they are worthy of citation - in general they should be peer-reviewed. 
Exceptions are seminal works, grey literature, or highly cited non-peer-reviewed sources. Make sure to evaluate the quality of a working paper, 
for example, by looking at other scientific contributions of the authors or the affiliated institution. 

\newthought{Citations} are essential in academic writing to give credit to original sources and avoid plagiarism. You have to include a reference to the original source in the text for every argument, source code, information, or line of thought that originates from another author.
Failing to do so will result in your work being classified as plagiarism and considered an attempt to deceive. 


  
  

\newthought{Narrow down} to the specific problem statement (in 2-4 bullet points)  or research question your thesis aims to address.
Then clearly (in 2-4 bullet points) mirror the problem statement and articulate the contributions you will be making and the significance of your work, 
and how the field will benefit from your work.\\

\marginnote{
\newthought{Schöpfungshöhe}, a certain level of originality and creativity required in academic work, is closely linked to the nature of your contributions. 
You will soon find yourself trading off between high-risk-high-impact ideas and more conservative incremental objectives. 
I advise to aim for a mix when defining your contributions: A workhorse (driven by execution and rigor), a staircase (small incremental improvement on a known method), 
a moonshot (high-impact idea or novel recombination, which might fail).
}

Conclude the introduction with a brief (really brief) outline of the thesis structure, guiding the reader on what to expect in the subsequent chapters.


\section{Methods}\index{Methods}

\newthought{The methods section} is $40 - 50\%$ of the total thesis length and is structured into 2-3 sections.
Describe the methodology used to address the research questions. 
Include all relevant mathematical formulations, algorithms, concepts and thoroughly explain the processes of data collection, generation, and analysis. 
Ensure that the description is sufficiently detailed to allow readers to reproduce your approach and achieve comparable results.

\begin{itemize}
  \item Research Design and Approach (usually new or adapted models/algorithms)
  \item Data Collection and Preparation or Description
  \item Experimental Setup and Implementation
  \item Evaluation Metrics and Analysis Methods
  \item Reproducibility and Validation
\end{itemize}

\newthought{Start with an overview} figure that depicts your overall architecture, flow and research design.  
Then proceed to describe the individual components in dedicated sections. The transition between sections need to be motivated and logical.

\newthought{Do not loose touch} with the problem statement and objectives defined in the introduction, make sure to clearly link back to them where appropriate.

\newthought{This chapter} contains the theoretical foundations, models, algorithms, experimental setups, data recorded, metrics, evaluation strategies, or analytical techniques employed in your research.

\marginnote{
  \newthought{Don't cite!}, reference. 
  \textit{The methods section outlines your work, not the work of others. Existing knowledge needs to be described and introduced in the Introduction. 
  If you feel the need to refresh the reader on established methods, do so by referencing back.}
}





\section{Results}\index{Results}
\newthought{The results section} is $25 - 35\%$ of the total thesis length and presents the findings of your research.
Results should be presented as objectively as possible, without mixing interpretation or discussion. 
Focus on reporting the findings clearly and concisely, using neutral language and avoiding subjective statements. 

\newthought{Always have several (at least two) benchmarks} to compare your results against. Great benchmarks are human performance, state-of-the-art methods,
established baselines, theoretical limits, or \textbf{heuristics}.


\marginnote{
  \newthought{Visual aids} such as tables, graphs, and charts are invaluable.
}

\newthought{Structure your results} by grouping related findings and presenting them in a logical order (some might call this story telling) 
that reflects your research objectives. 
Each result should be linked to the corresponding problem statement - again.

\begin{itemize}
  \item Present quantitative results with appropriate statistical measures (e.g., means, standard deviations, confidence intervals).
  \item Use tables to organize numerical data and figures to illustrate trends, patterns, or relationships.
  \item \textbf{Avoid interpreting or explaining} the implications of the results in this section; reserve such commentary for the discussion part.
\end{itemize}




\section{Discussion}\index{Discussion}

\newthought{The discussion section} is $15 - 20\%$ of the total thesis length and interprets and contextualizes your results, linking them back to your research questions and objectives.
A good practice is to make the linkage to your objectives explicit, by again using a bullet point list mirroring the objectives from the introduction.

\marginnote{
  \newthought{Conclusion}, think: Take-home message.\\
}

\newthought{Feel free to split} the \textit{Discussion} into a \textit{Discussion} and \textit{Conclusion} part. 

\newthought{The discussion} interprets and contextualizes your results and their limitations.
\newthought{While the conclusion} clearly distills the the main contributions of your work, reflects on implications, and points to future work.





\section{Appendices}\index{Appendix}
\newthought{The appendix} contains tables, data, questionnaires, proofs, derivations, and other ancillary information that might 
otherwise negatively affect the flow of the main text. 
The main text must reference the appendix where appropriate. 
Do not use the appendix for outsourcing text that does not fit into the main text due to page restrictions. 
Include an appendix only when necessary.






 
  

  
 





% ----------------------------------------------------------------------  
\chapter{Format}\index{Format}

\newthought{As you will be using a template} to write your thesis, you will be adhering to a predefined format already.
This includes specifications for margins, font size, line spacing, and citation style. But this only gets you so far.
\newthought{The last mile} is up to you, your attention to detail and your diligence. 
Sadly this is where many theses fail to deliver on a basic expectation, this is completely unnecessary - this is something to grind your way out.
\marginnote{Great formatting requires diligence, the good news is, this \textit{only} requires diligence.}

\section{On Length}\index{Length}
  \newthought{Brevity is valued} -so you should write as little as possible, as much as necessary.

  \marginnote{
    \textbf{On being concise:} \\
    Embrace short sentences and paragraphs, no fill-words. \\
    Equations and figures can condense information by a lot. \\
    Focus on the core aspects.}

  \newthought{As a rule of thumb, }
  \begin{itemize}
    \item A title or header should have no more than 13 words.
    \item A sentence should have around 15–20 words, maximum 25 words.
    \item A paragraph or subsection should ideally be around 5–7 sentences, maximum 10 sentences.
    \item A section should ideally be around 3–5 pages, maximum 7 pages.
    \item A chapter should ideally be around 6–12 pages, maximum 15 pages.
    \item A thesis should ideally be around 30–45 pages, maximum 60 pages.
  \end{itemize}

\section{On Hierarchy}\index{Hierarchy}
\marginnote{%
 1. Chapter \\
 2. Chapter \\
 2.1 Section \\
 2.1.1 Subsection \textbf{no, see a)} \\
 2.1.1.1 Subsubsection \textbf{no, see b)}  \\
 2.1.1.2 Subsubsection \textbf{no, see b)}  \\
 2.2 Section \\
 }
Subsections should be numbered using a decimal system that reflects their hierarchical position within the chapter (e.g., 1., 1.1, 1.2, 2.1, etc.).
If there is a need for further subdivision, use a third level of numbering (e.g., 1.1.1, 1.1.2).
Dividing only makes sense if,
\begin{enumerate}[label=\alph*)]
    \item you end up with at least two subsections when you divide a section.
    \item you avoid going deeper than three levels of hierarchy to maintain clarity.
    \item there is enough and distinct content to justify it.
\end{enumerate}




%----------------------------------------------------------------------
\section{Lists}\index{Lists}
\marginnote{%
  Keep each item of a bullet point list, \\
  \begin{itemize}
    \item readable on its own.
    \item parallel in structure.
  \end{itemize}
  \vspace{1em}

  Aphabetical lists make items,\\
  \begin{enumerate}[label=\alph*)]
    \item referencable.
    \item not necessarily ordered.
  \end{enumerate}  
  \vspace{1em}

  These items definitely,\\
  \begin{enumerate}[label=\arabic*.]
    \item follow a specific,
    \item sequence.
  \end{enumerate}
}
When using lists, whether numbered or bulleted, ensure they are formatted consistently throughout the thesis.
Use parallel structure for list items, meaning that each item should be readable as a standalone phrase or sentence.
Use numbered lists for sequences or steps that require a specific order, and bulleted lists for items that do not have a particular sequence.
Use alphabetic lists when you need to reference items individually within the text, but they do not have a specific sequence.
Use appropriate punctuation at the end of each list item, using commas or semicolons for items that are phrases, and periods for complete sentences.


%----------------------------------------------------------------------
\section{Figures and Tables}\index{Figures}\index{Tables}

Figures and tables should be numbered consecutively throughout the thesis for easy reference. 
They should be placed as close as possible to the first point of reference in the text - that also means there is at least one.
Each figure and table should have a clear and descriptive caption that explains its content and relevance. Keep it concise 2-3 sentences.
For further guidance on visualizing data effectively, refer to Tufte's principles on the visual display of quantitative information~\cite{TufteVisuals2001}.

\begin{figure}[!ht]
  \includegraphics[width=\linewidth]{figures/tufte-line.png}
  \caption{Basic line graph after Tufte. Data emphasized, axes info in title, minimal ticks, reduced non-data ink.}
  \label{fig:linegraph}
\end{figure}

\newthought{Tufte's Principles \cite{TufteVisuals2001} for Effective Figures (for comparison see Fig.~\ref{fig:linegraph}):}
\begin{enumerate}
    \item \textbf{Maximize the Data-Ink Ratio:} Use as little non-essential ink as possible (including colour usage). Every mark should represent data or support its understanding.
    \item \textbf{Erase Non-Data Ink (Chartjunk):} Remove decorative elements such as borders, heavy gridlines, 3D effects, shading, clip art, etc. If it doesn’t convey information, it shouldn’t be there.
    \item \textbf{Erase Redundant Data-Ink:} Don’t repeat information unnecessarily—avoid duplicate labels, overly dense tick marks, thick axes, repeated symbols. Keep only what’s needed.
    \item \textbf{Emphasize Clarity and Precision:} Be accurate, honest, and exact in visual encoding—use proportional scales, correct baselines, avoid distortion and misleading area/volume effects. Clarity over decoration.
    \item \textbf{Use Small Multiples:} Present multiple comparable graphics with consistent design. Useful for comparing groups, showing changes over time, or exploring multidimensional data.
    \item \textbf{Integrate Words, Numbers, and Graphics:} Place labels near the data, integrate text directly into the plot, and avoid legends unless unavoidable. Visual and textual explanations should coexist naturally.
\end{enumerate}




\newthought{When referencing} figures and tables in the text, use their assigned numbers (e.g., "as shown in Figure 2.1" or "see Table 3.2") - LaTex does that for you.
If you want to keep it short it is also acceptable to write (see Fig.~2.1) or (see Tab.~3.2).


\begin{table}[!ht]
  \centering
  \caption{Example of a well-formatted table with a clear, concise caption. Make sure to go \textbf{full width} for tables and figures, and embrace \textbf{horizontal lines}, while avoiding vertical lines.}
  \label{tab:example}
  \begin{tabular*}{\linewidth}{@{\extracolsep{\fill}}lcc}
    \toprule
    \textbf{Category} & \textbf{Value 1} & \textbf{Value 2} \\
    \midrule
    Example A & 42 & 3.14 \\
    Example B & 17 & 2.71 \\
    \bottomrule
  \end{tabular*}
\end{table}

\marginnote{\newthought{Descriptive} meaning, that it answers the question, 
\textit{What am I looking at and why is it important?}. Not, \textit{"on the x-Axis we have time in seconds}.\\
\vspace{1em}
  \textbf{A good example:} \\
    \textit{Figure 2.1: Overview of AI adoption trends across industries, illustrating the rapid increase in usage and highlighting key sectors driving innovation.} \\
  }

\newthought{Ensure} that all figures and tables,
\begin{itemize}
  \item reference source information if they are not original.
  \item are relevant to the content and contribute to the reader's understanding.
  \item are referenced in the text before they appear.
  \item are still interpretable in grayscale, when using colour.
\end{itemize} 



\section{Mathematical Notation and Equations}\index{Mathematical notation}\index{Equations}

\marginnote{%
  scalar variables: \(x, y, z\) \\
  vectors:  \(\mathbf{v}, \mathbf{u}\) \\
  matrices:  \(\mathbf{I}, \mathbf{A}\) \\
  sets: \(\mathcal{X}, \mathcal{D}\) \\
  functions: \(f(x)\) or \(\mathcal{L}\)\\
  constants: \(c, G, \pi\)) \\
  units: m, s, kg \\
  groundtruth:  \(\tilde{y}\) \\
  estimate/prediction: \(\hat{y}\) \\
  input data: \(x_i\) \\
  mean: \(\bar{y}\) \\
  cross product: \(\mathbf{a} \times \mathbf{b}\) \\
  dot product: \(\mathbf{a} \cdot \mathbf{b}\) \\
}

\newthought{Present mathematical notation} clearly and consistently. Define all symbols and units on first use. Often it is helpful to include the equation into a sentence,
such as "The relationship between force $\textbf{F}$, mass $m$, and acceleration $a$ is given by Newton's second law, expressed as:
\begin{equation}
  \textbf{F} = m \cdot a.
\end{equation}
Notice how the equation is part of the sentence and ends with a period. When embedded into a sentence use a comma if the sentence continues after the equation.
When displaying equations, center them on the page and number them consecutively for easy reference (e.g., Eq.~1, Eq.~2).


\section{Code}\index{Code}
  \label{sec:code}
  \marginnote{%
    \textbf{Runs on my machine.} \\
    Is not acceptable, make sure to host your solution (or a meaningful part of it). At least provide a Docker container or similar means to reproduce your results fast and reliably.
  }

  \newthought{When including code} in your thesis, it will be pseudo-code.
  For everything else use proper references to code repositories, e.g., GitHub, GitLab, Bitbucket.
  Ensure that any code snippets included are well-documented and formatted for readability.

  \newthought{Code plays a vital role} in your research. Think of code as a means to communicate, make your contributions transparent, reproducible, and extendable for future studies.
  Thus it is best practice to host your code in a public repository,
  providing a clear README file, usage instructions, and documentation of dependencies and environment setup.



  \begin{algorithm}[H]
  \caption{Pseudo-code for Gradient Descent Optimization}
  \begin{algorithmic}[1]
    \Require Initial parameters $\theta_0$, learning rate $\alpha$, loss function $L(\theta)$
    \State $t \gets 0$
    \While{not converged}
      \State Compute gradient: $g_t \gets \nabla_\theta L(\theta_t)$
      \State Update parameters: $\theta_{t+1} \gets \theta_t - \alpha g_t$
      \State $t \gets t + 1$
    \EndWhile
    \Return $\theta_t$
  \end{algorithmic}
  \end{algorithm}

\section{Abbreviations}\index{Abbreviations}
\newthought{Abbreviations} need to be defined upon their first use in the text.
\marginnote{%
  For example, \textit{Artificial Intelligence (AI)}. \\
  Thereafter, the abbreviation can be used throughout the document. \\
}
Ensure that abbreviations are used consistently and appropriately, avoiding overuse which can hinder readability.
In titles, headings, and figure and table captions, abbreviations are not to be used. 



\section{Punctuation and Signs}\index{Punctuation}\index{Signs}
\newthought{Punctuation} in English clarifies meaning and aids readability. 
\marginnote{
\textbf{Examples:}
\begin{itemize}
  \item Capital: \textit{Dr David James is the consultant at Leeds City Hospital.}
  \item Full stop: \textit{We went to France last summer.}
  \item Question mark: \textit{Why do they make so many mistakes?}
  \item Exclamation: \textit{Listen!}
  \item Comma: \textit{It’s important to write in clear, simple, accurate words.}
  \item Colon: \textit{There are three main reasons: economic, social, political.}
  \item Semi-colon: \textit{Spanish is spoken throughout South America; in Brazil the main language is Portuguese.}
  \item Quotation: \textit{She said, “Where can we find a nice Indian restaurant?”}
  \item Dash: \textit{Our teacher – who often gets cross – wasn’t cross at all.}
  \item Brackets: \textit{Thriplow (pronounced ‘Triplow’) is a small village.}
  \item Numerals: \textit{7,980} (thousands), \textit{6.5} (decimal)
\end{itemize}
}

Key marks include: capital letters (start sentences, proper nouns), 
full stops (end sentences), question marks (indicate questions), 
exclamation marks (emphasis, in doubt don't), commas (lists, clauses, speech), colons (introduce lists/explanations), 
semi-colons (link related clauses), quotation marks (direct speech, highlight words), 
dashes/brackets (extra info), and numerals (full stops for decimals, commas for thousands)~\cite{CambridgePunctuation}.

Punctuation marks should be placed inside quotation marks if they are part of the quoted material, and outside if they are not.
In terms of visual appearance, punctuation and other signs can not be at the beginning of a line, bind signs to a word by making use of \textit{~} in LaTeX.

\section{Line and Page Breaks}\index{Line breaks}\index{Page breaks}
\newthought{Widows and orphans} are to be avoided at all cost.
A widow is a single word or short line that appears at the end of a paragraph and is left alone at the top of a new page, after a table or figure or column.
An orphan is a single word or short line that appears at the beginning of a paragraph and is left alone.
To prevent widows and orphans, adjust the text by rewriting sentences or changing spacing, e.g. \verb|\vspace|.   







% ----------------------------------------------------------------------
\chapter{Writing}\index{Writing}

\newthought{Write with the reader in mind.}
\marginnote{%
  \newthought{A thesis} is usually not read in linear sequence. But rather Abstract - Problem Statement - Conclusion.\\
  \textit{And only then will the inclined reader go for the methods and results sections in depth. Make sure to not loose them on the way.}
}


\newthought{Good scientific writing} is an art form - it is clear, specific, concise, and coherent. 
An indepth treatment of style and writing is delivered by Steven Pinker \cite{PinkerSenseOfStyle2014}

\section{Be Clear}\index{Clarity}
It is crucial to establish a central idea or line of thought that runs throughout your thesis. 
This theme should be reflected in the title, table of contents, and in the relevance of every sentence and section to your main research question. 
Always make clear to the reader how each part of your work connects back to your problem statements.

\marginnote{\newthought{When referencing your own} chapters or sections, make use of \texttt{\textbackslash ref\{sectionname\}}, avoid forward references.}

Explicitly discuss links between different results and highlight the contributions of your work connecting back to your problem statements.
If any word, sentence, paragraph, or section does not support your central theme or research question, it should be omitted.



\section{Be Specific}\index{Specificity}
Always argue based on results and sources - and whenever possible quantify. 
Avoid vague expressions such as \textit{"the results are good ..."} or \textit{"I feel that ...”}.
Also avoid imprecise statements such as \textit{"Under certain assumptions, we get the following result ..."} or \textit{“It is generally known that ..."}. 
Of course, you must document any facts, claims and opinions that you have borrowed from the literature. 
Nevertheless, when leaning on sources, make sure to give them context, and elaborate on their essence to enable a clearly structured argument.

\section{Be Concise}\index{Conciseness}
Each sentence should convey a single thought clearly and directly. Each paragraph should develop a complete idea \cite{FreiburgThesisGuideline2025}.

  \begin{itemize}
    \item Use active voice where possible.
    \item Eliminate redundant phrases (e.g., "in order to" → "to").
    \item Prefer simple words over complex ones (e.g., "use" instead of "utilize").
    \item Break long sentences into shorter ones.
    \item Avoid filler words (e.g., "very," "really," "just").
    \item Avoid "we" or "I" unless necessary.
    \item Ensure orthographic and grammatical correctness!
  \end{itemize}


\section{Be Consistent}\index{Consistency}
Consistency is key throughout everything in your thesis. You decided for a specific word choice, a specific notation, a specific nomenclature - stick to it.

\newthought{Especially} when you decided on a specific terminology. Avoid synonyms to achieve creative variation, such as drug, medication, etc. in favour
of semantic consistency for the reader.








\chapter{Concluding}\index{Conclusion}
\newthought{Writing a scientific thesis is a process} that demands clarity, structure, and diligence. 
By following established best practices, working smart and maintaining academic integrity 
you can produce a thesis that is both impactful and accessible. 
Remember, the journey is iterative—embrace feedback, stay organized, and let your work reflect expertise, craftsmanship, and thoughtful originality. 

\newthought{You can look forward to scrolling through your thesis,} it will be oddly satisfying.






\makestandardbackmatter{howtothesis_references}

\end{document}
