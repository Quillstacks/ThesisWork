
\section{Timeline}
\newthought{Begin with a concise overview of your project’s current status} compared to the plan you outlined earlier.
Take this opportunity to reflect on your objectives and timeline, and discuss any adjustments you have made or need to make.
You can use Table~\ref{tab:timeline} as a reference for structuring your updated timeline. Ensure the descriptions are actionable and specific to your work. 
The table does not count toward your page limit, so provide as much detail as needed.

\begin{table}[ht]
  \centering
  \begin{tabular}{p{0.23\linewidth} p{0.55\linewidth} p{0.17\linewidth}}
    \toprule
    \textbf{Milestone} & \textbf{Description} & \textbf{Timeline} \\
    \midrule
    Review & Survey recent and seminal works, identify gaps and relevant ressources. & Week 1 \\
    Definition & Formulate research questions and hypotheses based on literature findings. & Week 2 \\
    Development & Prepare data, develop methods, and conduct experiments. & Week 3--9 \\
    Evaluation & Analyze results, compare with baselines, and interpret findings. & Week 7--11 \\
    Submission & Document methods, results and conclusions; polish for thesis submission. & Week 10--12 \\
    \bottomrule
  \end{tabular}
  \caption{Project timeline and milestones for a 12-week research project (e.g. Bachelor Thesis).}
  \label{tab:timeline}
\end{table}



\section{Overview Schematic}

\newthought{Start with an overview} that depicts your overall system, architecture, framework, and process flows.
Think of this as the blueprint for your methods chapter - again this exercise is a means to an end. 
My doctoral advisor used to call this schematic \textit{the thesis in one slide}.
The different components and their interactions should be clearly outlined. Where possible these also map back to your objectives and prefigure your final contributions. 

\newthought{With the overview schematic in mind,} proceed to describe each component on a high level in text as well. 







\section{Proof of Concept}
\marginnote{\newthought{This is the place}, for visuals, equations, or pseudo-code snippets that illustrate your approach.}

\newthought{Introduce the prototype, proof of concept or initial implementation work} you have completed. 
Demonastrate development activity by providing a deployment or a means to deploy and reproduce your work.
Think jupyter notebook, think Docker container, think API endpoints, think Demo Video - whatever suits your project best, 
and make sure to provide access.
This may also include pseudo code, initial experimental results, data analysis, or design of experiments, which you should document here.

\begin{algorithm}[H]
  \caption{Pseudo-code for Gradient Descent Optimization. If this is not beautiful, I don't know what is.}
  \label{alg:gradient_descent}
  \begin{algorithmic}[1]
    \Require Initial parameters $\theta_0$, learning rate $\epsilon$, loss function $L(\theta)$
    \State $t \gets 0$
    \While{not converged}
      \State Compute gradient: $g_t \gets \nabla_\theta L(\theta_t)$
      \State Update parameters: $\theta_{t+1} \gets \theta_t - \epsilon g_t$
      \State $t \gets t + 1$
    \EndWhile
    \Return $\theta_t$
  \end{algorithmic}
\end{algorithm}

\newthought{Finish this section} by summarizing the key functionalities of your prototype or initial implementation, 
and which aspects are still under development or pending validation.




