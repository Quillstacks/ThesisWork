%%%%%%%%%%%%%%%%%%%%%%%%%%%%%%%%%%%%%%%%%%%%%%%%%%%%%%%%%%%%%%%%%%%%%%
% Expendable Thesis Template
%
% Author: Dr.-Ing. Mark Schutera
% Based on Tufte-style layout
% 
% This is the document to select when running your LaTeX build process
% When in doubt, run: pdflatex -> bibtex -> pdflatex -> pdflatex
% When still in doupt delete auxiliary files and try again
\makeindex
%%%%%%%%%%%%%%%%%%%%%%%%%%%%%%%%%%%%%%%%%%%%%%%%%%%%%%%%%%%%%%%%%%%%%%

\documentclass{tufte}

%%
% Book metadata
\title{Expendable Thesis Template}
\author[\defaultauthor]{\defaultauthor}
\publisher[\defaultpublisher]{\defaultpublisher}

% Load optional fonts
\loadoptionalfonts

% Additional packages
\usepackage{pifont}
\usepackage{amsmath}
\usepackage{soul}
\usepackage{textcase}
\usepackage{enumitem}
\usepackage{tikz}
\usepackage{pgfplots}
\usepackage{float}
\usepackage{algorithm}
\usepackage{algpseudocode}
\usepackage{lipsum}
\setcitestyle{numbers,square}



\begin{document}
\makestandardfrontmatter
% Dedication
% Alternative quotes:
\makededication{\openepigraph{“We're just clones, sir. We're meant to be expendable”}{Clone Trooper of the 104th Battalion}}



% Introduction
\makeintroduction{
  We have all been there - facing that empty stack of papers or pixels for that matter, ultimately facing the daunting task of writing a scientific thesis.
  \newthought{This template} tries to bypass some of that empty paper anxiety by providing some intertia to get you going. 
  Feel free to clone and adapt it to your needs. If you come across parts which you adapt, and think should be adapted in the template in general, 

  \newthought{Please Contribute back - } This project is open for contributions, 
  via \href{https://github.com/Quillstacks/ThesisWork/tree/952300e7be221eb82c4f1b615db7cd3ace1ce774/ThesisTemplate}{GitHub} to report issues and suggest improvements and changes, submit a pull requests.


  \newthought{While this template} provides implicit guidance on structure, format and style, it is not a substitute 
  for reading the companion guide.
  
  \marginnote{\newthought{To get} to the guide click 
  \href{https://pages.schutera.com/HowToThesis/notes.pdf}{here}
  }

  \newthought{This document is a living work and should not be considered complete}—there are approaches I have tested and refined, 
  others I am still exploring, some I will adapt based on new insights, and perspectives I thought I understood but continue to evolve.
  That being said, in a near future this space will be ..

  \vfill
  \newthought{Your Abstract.} The abstract provides a concise summary of your thesis, including the research question, 
  chosen approach, key findings, and their interpretation. 
  It should be no longer than half a page. While optional, feel free to write the abstract in English 
  and the native language of your institution.
  The abstract is best written after the main sections are complete, 
  ensuring it accurately reflects the content and contributions of your work. This is probably the only part of your thesis that will be read by everyone,
  so make it count - and make the reader understand why to read on.
  }




  \makeacknowledgment{%
  \newthought{You will want to thank} your supervisor(s), your industry partners and institute, colleagues, friends, family and anyone else who supported 
  you during your journey. If you want to get personal and lyric, this is the place in your thesis to do so. 
  \newthought{For my part,} I want to express my sincere gratitude to all those who have and will have contributed to the development and refinement of this thesis template.
  \newthought{And I want to thank my wife,} for putting up with me stealing weekends and evenings in exchange for this template. 
  }%

  \makeaffirmation

  
  \cleardoublepage
  \tableofcontents




%% Start the main matter
\mainmatter


% ----------------------------------------------------------------------

\chapter{Introduction} \index{Introduction}
\label{ch:Introduction}
\newthought{Every Chapter}, starts with a block of text, which introduces the chapter and its role in the overall thesis structure. 
At the same time this block of text opens up potential sections within the chapter. As for the introduction this first block takes care 
of the context and motivation, of your thesis. The introduction is $20 - 30\%$ of the total thesis length. 

\marginnote{\newthought{Lengthy motivation sections}\\
  covering how autonomous driving brings down traffic accidents by 90\%, reiteration of other broad concepts, 
  or humble bragging in favor of your industrial partner, or similar topics, are to be avoided - A short single sentence will do.
  Then focus on motivating your specific field and research problem, this is in its own right a broad field.
}

\section{Problem Statement}\label{sec:ProblemStatement}
\newthought{Why is this specific field of research important, now?} Then, quickly narrow down to the specific field you will be addressing.
Broadly outline the current state of the art in this field, meaning all relevant fields for that matter, and how they relate to reach other. Be diligent in preparing the ground for pointing out
gaps you identified in existing knowledge or limitations in current approaches and the overarching problems that arise from the field. 
These gaps are often a combination of several factors, such as limitations of current approaches, gaps in theoretical knowledge, constraints in applications, 
or emerging developments that necessitate further investigation. 

\newthought{Citations} are essential in academic writing to give credit to original sources. 
In your proposal, there is no reference section, instead they will all show up in the marginnotes like so~\cite{SchuteraBachelorThesis}.
This is achieved by use of the \texttt{\textbackslash cite\{\}} command. This is also means that you should limit and focus yourself
to the most relevant works. Later in your thesis you will have a full reference section at the end of your document, and this section will be brimming with citations.

\vfill
\section{Objectives}\label{sec:Objectives}
\marginnote{%
\newthought{Schöpfungshöhe}, a certain level of originality and creativity required in academic work, is closely linked to the nature of your contributions. 
You will soon find yourself trading off between high-risk-high-impact ideas and more conservative incremental objectives. 
I advise to aim for a mix when defining your contributions: A workhorse (driven by execution and rigor), a staircase (small incremental improvement on a known method), 
a moonshot (high-impact idea or novel recombination, which might fail).
} 

\newthought{Open this} with a 3-5 sentences which distill the derived problem statement. 
Then clearly (in 3-5 bullet points) outline the problem statement and articulate the contributions you will be making. 

\newthought{The contributions} should be:

\begin{itemize}
  \item Specific: Clearly define what you aim to contribute.
  \item Measurable: Ensure that your contributions can be evaluated.
  \item Achievable: Set goals that are realistic to be accomplished.
  \item Relevant: Align your contributions with the problem statement.
  \item Time-bound: Specify a timeline for achieving each contribution.
\end{itemize}
\marginnote{
\newthought{Specific contributions}, could sound like: A new method for X, More sensitive metrics for Y, an empirical study on Z, a curated dataset for X, 
a data-driven analysis of Y, or an application of Z for a new domain. After reading the 
bullet points, the reader should be able to clearly understand what you are trying to achieve and what he will be walking away with.
}


\section{Timeline}
\newthought{Present a timeline} that outlines the key milestones and deadlines for your research project.
You can make use of Table~\ref{tab:timeline} as a reference for structuring your own timeline. Make sure to 
update the descriptions to make them actionable for your work. The table does not count into your page limit, 
you will benefit from greater detail here.

\begin{table}[ht]
  \centering
  \begin{tabular}{p{0.23\linewidth} p{0.55\linewidth} p{0.17\linewidth}}
    \toprule
    \textbf{Milestone} & \textbf{Description} & \textbf{Timeline} \\
    \midrule
    Review & Survey recent and seminal works, identify gaps and relevant ressources. & Week 1 \\
    Definition* & Formulate research questions and hypotheses based on literature findings. & Week 2 \\
    Development & Prepare data, develop methods, and conduct experiments. & Week 3--9 \\
    Evaluation & Analyze results, compare with baselines, and interpret findings. & Week 7--11 \\
    Submission & Document methods, results and conclusions; polish for thesis submission. & Week 10--12 \\
    \bottomrule
  \end{tabular}
  \caption{Project timeline and milestones for a 12-week research project (e.g. Bachelor Thesis).}
  \label{tab:timeline}
\end{table}

\marginnote{\newthought{*With this submission} you complete the second milestone.} 




\chapter{Methods} \index{Methods}
\label{ch:Methods}


\newthought{The methods section} is structured into sections reflecting the different aspects of your research.
These sections initiate your novel approaches, concepts, designs, algorithms, metrics, evaluation strategies, analysis methods, and frameworks 
used to address the research objectives.
Include all relevant mathematical formulations, algorithms, hyperparameters and procedural steps necessary to understand your work.
Ensure that the description is sufficiently detailed to allow readers to reproduce your approach and achieve comparable results.


\marginnote{%
  \newthought{This Chapter is the place} for visuals, schematics, pseudo code and mathematical equations.
}


\newthought{Start with an overview} that depicts your overall system, architecture, framework, and process flows.
Think of this as the blueprint for your methods chapter. My doctoral advisor used to call this schematic \textit{the thesis in one slide}.
The different components and their interactions should be clearly outlined. 
It is nifty to repurpose the visuals of the components in the following sections, adding more detail there, while referencing back to the overview.

% You want to contribute? This is the place you are looking for. 
% I am still trying to find a good library to draw nice system diagrams in LaTeX, or at least include them nicely.
% Bonus points using Tufte Style (maybe that package or library still needs to be written, maybe we should do that).
% Getting this step standardized across theses would be great. Reach out if you think this is something you want to work on.

\newthought{Where possible} these components can also map back to or even mirror the objectives defined in the Introduction.
In any case do not loose touch with the problem statement and make sure to have it in mind when writing this Chapter and its sections.

\marginnote{%
  \newthought{Don't cite!}, reference. 
  The methods section outlines your work, not the work of others. Existing knowledge needs to be described and introduced in the Introduction (see Sec.~\ref{sec:RelatedWork}). 
  If you feel the need to refresh the reader on established methods, do so by referencing back.
}

\newthought{After the overview,} proceed to describe each component on a high level in text as well. If you are designing an entire new framework or system. 
This is also the place to introduce new definitions, notations, or terminology that will be used throughout the rest of the thesis.


\section{First Component}
\newthought{Then proceed to describe} the individual components in dedicated sections. 

\textit{Remember you wanted to reuse parts of the overview visuals here.}

The transition between sections need to be motivated and logical, often advancements in one component bring about needs, 
which are then adressed in following components. For example you might have a new data collection method, which in turn enables a new optimal model architecture,

\begin{equation}
\hat{y} = \theta^*(x),
\end{equation}

which in turn requires a new evaluation metric. And so forth.
\marginnote{%
  \newthought{Make use of margin notes} to provide ancillary definitions, extended explanations, walkthrough examples, or additional context which would otherwise disrupt the main narrative.
  Things that are nice to know, or support the reader, but which are not essential to the core understanding.
}
Of course this is not always that linear, but try to keep a logical flow - branching out is expected.

\section{Second Component}
\newthought{The motivation of a new section} should be in the beginning of that very section, 
so that the section itself is self-contained. 

\textit{Remember you wanted to reuse parts of the overview visuals here.}

This Chapter is usually where the beauty (see for yourself in Alg.~\ref{alg:gradient_descent}) of your work is to be found, having honed and refined your approaches over long hours. 


\begin{algorithm}[H]
  \caption{Pseudo-code for Gradient Descent Optimization. If this is not beautiful, I don't know what is.}
  \label{alg:gradient_descent}
  \begin{algorithmic}[1]
    \Require Initial parameters $\theta_0$, learning rate $\epsilon$, loss function $L(\theta)$
    \State $t \gets 0$
    \While{not converged}
      \State Compute gradient: $g_t \gets \nabla_\theta L(\theta_t)$
      \State Update parameters: $\theta_{t+1} \gets \theta_t - \epsilon g_t$
      \State $t \gets t + 1$
    \EndWhile
    \Return $\theta_t$
  \end{algorithmic}
\end{algorithm}


This is where you demonstrate your creativity, and problem-solving abilities.




\chapter{Results} \index{Results}
\label{ch:Results}
\newthought{The results section} introduces experimental setups, experimental design, datasets used, hardware, implementation details, 
and presents the findings of your experiments and analyses around your novel methods - in a quantitative manner. Results should be presented as objectively as possible, without blending in interpretations or discussions at this point. 
Focus on reporting the findings clearly and concisely, using neutral language and avoiding subjective statements.

\begin{table}[!ht]
  \centering
  \caption{Example of a well-formatted table with a clear, concise caption. Make sure to go \textit{full width} for tables and figures, and embrace \textit{horizontal lines}, while avoiding vertical lines.}
  \label{tab:example}
  \begin{tabular*}{\linewidth}{@{\extracolsep{\fill}}lcc}
    \toprule
    \textbf{Category} & \textbf{Value 1} & \textbf{Value 2} \\
    \midrule
    Example A & \textit{42} & 3.14 \\
    Example B & 17 & 2.71 \\
    \bottomrule
  \end{tabular*}
\end{table}

 

\newthought{Always have several (at least two) benchmarks or baselines} to compare your results against. Great baselines are human performance, 
state-of-the-art methods, other established baselines, theoretical limits, or heuristics and simple models for starters. 
Present quantitative results with appropriate statistical measures (e.g., means, standard deviations, confidence intervals).
Deciding what and how to measure is crucial. Aligning your metrics and evaluation criteria with your research objectives is a field of study in its own right.
Whenever possible, use established benchmarks and metrics from the literature (you will have introduced them in your Introduction, in Section~\ref{sec:RelatedWork}) to ensure comparability. 
New developments and designs in this are shows in the Methods Chapter (see Ch.~\ref{ch:Methods}).

\newthought{Use tables to organize} numerical data (see Tab.~\ref{tab:example}) and figures (see Fig.~\ref{fig:linegraph}) to illustrate trends, patterns, or relationships.

% You want to contribute? This is the place you are looking for. 
% I am still trying to find a good library to draw nice figures and plots in LaTeX, or at least include them nicely.
% Bonus points using Tufte Style (maybe that package or library still needs to be written, maybe we should do that).
% Getting this step standardized across theses would be great. Reach out if you think this is something you want to work on.

\begin{figure}[!ht]
  \includegraphics[width=\linewidth]{figures/tufte-line.png}
  \caption{Basic line graph after Tufte. Data emphasized, axes info in title, minimal ticks, reduced non-data ink.}
  \label{fig:linegraph}
\end{figure}


\section{Experimental Design}
\newthought{If you have non-trivial} experimental setups or designs, describe them here. 
This includes the design of experiments used to evaluate your methods.

\section{Implementation Details}
\newthought{This also holds} for details about datasets, hardware, software frameworks, and implementation specifics.
Again, focus on the things that are necessary to understand and reproduce your results - not every package you installed is relevant.
A lot will be documented in your code repository and code documentation anyway, so focus on the important bits here. 
And remember to use margin notes for ancillary information - this is especially powerful in this chapter.

\newthought{A good practice} is to have a code demo ready to showcase your results interactively. This could be a Jupyter notebook, 
a web application, or any other way of hosting software in this century that allows users to engage with your findings and contributions.
Make sure that this demo is properly documented and accessible, ideally set up for easy deployment (better application).
\marginnote{
  \newthought{When working on proprietary} or sensitive data, make use of a toy dataset or environment.
}














\section{First Finding}
\newthought{Structure your results} by grouping related findings and presenting them in a logical order (some might call this story telling) 
that reflects your research objectives. 
\marginnote{
  \newthought{Visual aids} such as images, technical drawings, tables, graphs, and charts are invaluable. 
  Yet, make sure the text itself can transfer the results self-sustained.
}
Each result should be linkable to one or multiple corresponding problem statements.






\section{Second Finding}
\newthought{Every method} presented in the Methods chapter should have corresponding results here.
Often your methods are evaluated along multiple dimensions, such as performance metrics, qualitative observations, and comparative analyses. 


\chapter{Discussion} \index{Discussion}
\label{ch:Discussion}
\newthought{The discussion section} is the place to interpret and contextualize your results, linking them amongst each other. 
Usually things become easier when branching out the \textit{Conclusion} part together with the \textit{Future Research} part. 
While in the Chapter on Results (see Ch.~\ref{ch:Results}) you presented your findings and experiments more or less isolated, and in neutral objectivity, 
this Chapter helps the reader to make meaning out of them. The meaning arises from putting your results in context against benchmarks and baselines. 

\newthought{Short Excursion on Benchmarks and Baselines.}
Baselines, are simpler or previously existing approaches that serve as a starting point for comparison. Such as heuristics, or human performance. 
Benchmarks, on the other hand, are established standards or reference points against which your results can be compared. 
They often represent the best-known performance or widely accepted methods in your field and usually provide means to test against them. The quantitative
calculations are of course done already in the Results Chapter, but the interpretation and discussion happens here.


\marginnote{%
  \newthought{This is the time}, to show-off the deltas you worked so hard for. 
}

\newthought{By comparing your results} against benchmarks and baselines, you can demonstrate and discuss the significance and limitations of your work.
While this is a quantitative exercise at first, the discussion allows for thoughtful qualitative interpretation. 
However, be careful to not over-interpret your results, nor is this the moment to introduce new ideas or hypotheses. 
Another important aspect of the discussion is to openly 
address limitations, white spots, and potential sources of error, meaning expected variances and biases, in your study. 







% ----------------------------------------------------------------------
\section{Conclusion}\label{sec:Conclusion}
\newthought{The conclusion section} serves to succinctly summarize the main contributions of your thesis work.
Clearly distill the main contributions of your work, and reflect on their implications. 
While the discussion section interprets the results in detail, the conclusion is more about showing the reader what they walk away with.


% A contribution ------------------------------------------------------
\newthought{For each objective,} in your Section~\ref{sec:Objectives}, mirror your contributions in dedicated blocks here.
Begin by restating the objective in a single sentence, followed by elaborating on how your methods and results addressed it.


% A contribution ------------------------------------------------------
\newthought{Remember,} this could be a new algorithm, model, framework, dataset, or empirical finding - or something in those lines.


% ----------------------------------------------------------------------
\newthought{Concluding,} make the linkage to your objectives explicit, by using a bullet point list mirroring the objectives from Section~\ref{sec:Objectives}, again.
Start out with a key statement, which summarizes the contribution when taking a holistic view: 
\begin{itemize}
  \item Then, for each objective, briefly discuss how your results contribute to it. Make it one to three sentences per contribution.
  \item If you have artifacts (code, data, models) to share, reference them here and provide permanent accessibility.
\end{itemize}








% ----------------------------------------------------------------------
\section{Recommendations for Future Research}\label{sec:FutureResearch}

\marginnote{%
  \newthought{It is a good habit}, to treat this section like a backlog. Documenting work packages which 
    have been in scope of this thesis, but exceeded the available time or resources. 
    Of course you need to prioritize this backlog before drafting this section.
}

\newthought{Based on the findings and limitations discussed}, outline potential avenues for future research that could build upon your work.
This could include exploring unanswered questions, testing your methods in different contexts, or addressing limitations identified in your study. 
While writing this, keep the next student or researcher in mind who might pick up where you left off 
- knowing everything you know now, what would you recommend them to include in their problem statement (compare with yours in Sec.~\ref{sec:ProblemStatement})?  


% ==========================================================================================================

\makestandardbackmatter{references} 

\chapter{Appendices}

\newthought{Appendix 1}
\label{app:1}
and a short description of appendix 1. This description is between 2-3 sentences long. The main text must reference the appendix where appropriate.
\newthought{Appendix 2}
\label{app:2}
contains tables, data, questionnaires, proofs, derivations, and other ancillary information that might 
otherwise negatively affect the flow of the main text.


\vfill
\newthought{The appendix} 
is not a dumping ground for material that does not fit into the main text due to page limitations.
Appendices are optional, include an appendix only when necessary. Often enough you do not need it.
\cleardoublepage


\end{document}
