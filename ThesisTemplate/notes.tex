%%%%%%%%%%%%%%%%%%%%%%%%%%%%%%%%%%%%%%%%%%%%%%%%%%%%%%%%%%%%%%%%%%%%%%
% How to Write a Scientific Thesis
% Author: Dr.-Ing. Mark Schutera
% Based on Tufte-style layout
\makeindex
%%%%%%%%%%%%%%%%%%%%%%%%%%%%%%%%%%%%%%%%%%%%%%%%%%%%%%%%%%%%%%%%%%%%%%

\documentclass{../sharedAssets/tufte}

%%
% Book metadata
\title{A Thesis Template}
\author[\defaultauthor]{\defaultauthor}
\publisher[\defaultpublisher]{\defaultpublisher}

% Load optional fonts
\loadoptionalfonts

% Additional packages
\usepackage{pifont}
\usepackage{amsmath}
\usepackage{soul}
\usepackage{textcase}
\usepackage{enumitem}
\usepackage{tikz}
\usepackage{pgfplots}
\usepackage{float}
\usepackage{algorithm}
\usepackage{algpseudocode}

\begin{document}



\makestandardfrontmatter
% Dedication
% Alternative quotes:
\makededication{\openepigraph{“We're just clones, sir. We're meant to be expendable”}{Clone Trooper of the 104th Battalion}}
\tableofcontents


% Introduction
\makeintroduction{%
  this is to be done
  }


\vfill
\par\smallcaps{Contribute - } This project is open for contributions, 
via GitHub \url{https://github.com/Quillstacks/ThesisWork} to report issues and suggest improvements, submit a pull requests.



%% Start the main matter
\mainmatter


% ----------------------------------------------------------------------

\chapter{Introduction}\index{Thesis writing}

\chapter{Methods}

\chapter{Results}

\chapter{Discussion}
\chapter{Conclusion}

% ----------------------------------------------------------------------

\chapter{Structure and Scope}\index{Structure}

\marginnote{%
  \newthought{Common Thesis Structure:} \\
  1. Title Page \\
  2. Abstract \\
  3. Affirmation \\
  4. Acknowledgements \\
  5. Table of Contents \\
  6. List of Figures and Tables \\
  7. List of Equations \\
  8. List of Abbreviations \\
  9. List of Symbols \\
  10. IMRD Body \\
  11. Index \\
  12. References \\
  13. Appendices \\
}

\newthought{The structure is for the reader.} You read that right, the structure is not primarily for you. 
Of course a good structure helps you to organize and document your thoughts and work as you go - but it would be no good advice to limit yourself to that.
Ultimately then, a well-organized thesis guides the reader through your research journey, 
helping them understand your objectives, methods, findings, contributions and conclusions.\\

\newthought{While some flexibility} is allowed and even encouraged to adapt the structure to your specific research topic and methodology,
the following sections outline a commonly accepted structure for scientific theses. The IMRD (Introduction, Methods, Results, Discussion) is a good starting point.

\section{Abstract}\index{Abstract}
\newthought{The abstract} provides a concise summary of your thesis, including the research question, 
chosen approach, key findings, and their interpretation. 
It should be no longer than half a page. While optional, feel free to write the abstract in English 
and the native language of your institution.
The abstract is best written after the main sections are complete, 
ensuring it accurately reflects the content and contributions of your work.

\section{Introduction}\index{Introduction}
\newthought{The introduction} is $20 - 30\%$ of the total thesis length and is structured into: 
\begin{itemize}
  \item Context and Motivation
  \item Related Work and Background through Literature Review
  \item Research Gap and Problem Statement
  \item Objectives and their Significance
  \item Thesis Structure Overview
\end{itemize}

\marginnote{\newthought{Lengthy motivation sections}\\
  covering how autonomous driving brings down traffic accidents by 90\%, or reiteration of broad concepts, 
  or (humble bragging around) how awesome your industrial partner is, or similar topics, are to be avoided - A short single sentence will do.
  Then focus on motivating your specific field and research problem.
}

Begin with a broad yet short overview to provide context and motivation in the current field.

\newthought{Related Work -} or State of the Art, summarizes and connects relevant literature to highlight and organize existing knowledge and identify \textbf{gaps}. 
Sometimes this is extended to a section on preliminary, or foundational concepts - however when you go that path, don't put too much into it. 
Make use of Google Scholar as a tool to do so.
\marginnote{%
  \newthought{Gaps can be diverse} \\
    Inconsistencies or contradictions in existing studies. \\
    Methodological limitations in previous research\\
    Missing transfer of methods between fields of application. 
}
Select your sources carefully and evaluate whether they are worthy of citation - in general they should be peer-reviewed. 
Exceptions are seminal works, grey literature, or highly cited non-peer-reviewed sources. Make sure to evaluate the quality of a working paper, 
for example, by looking at other scientific contributions of the authors or the affiliated institution. 

\newthought{Citations} are essential in academic writing to give credit to original sources and avoid plagiarism. You have to include a reference to the original source in the text for every argument, source code, information, or line of thought that originates from another author.
Failing to do so will result in your work being classified as plagiarism and considered an attempt to deceive. 

\newthought{Narrow down} to the specific problem statement (in 2-4 bullet points)  or research question your thesis aims to address.
Then clearly (in 2-4 bullet points) mirror the problem statement and articulate the contributions you will be making and the significance of your work, 
and how the field will benefit from your work.\\

\marginnote{%
\newthought{Schöpfungshöhe}, a certain level of originality and creativity required in academic work, is closely linked to the nature of your contributions. 
You will soon find yourself trading off between high-risk-high-impact ideas and more conservative incremental objectives. 
I advise to aim for a mix when defining your contributions: A workhorse (driven by execution and rigor), a staircase (small incremental improvement on a known method), 
a moonshot (high-impact idea or novel recombination, which might fail).
}

Conclude the introduction with a brief (really brief) outline of the thesis structure, guiding the reader on what to expect in the subsequent chapters.

\section{Methods}\index{Methods}

\newthought{The methods section} is $40 - 50\%$ of the total thesis length and is structured into 2-3 sections.
Describe the methodology used to address the research questions. 
Include all relevant mathematical formulations, algorithms, concepts and thoroughly explain the processes of data collection, generation, and analysis. 
Ensure that the description is sufficiently detailed to allow readers to reproduce your approach and achieve comparable results.

\begin{itemize}
  \item Research Design and Approach (usually new or adapted models/algorithms)
  \item Data Collection and Preparation or Description
  \item Experimental Setup and Implementation
  \item Evaluation Metrics and Analysis Methods
  \item Reproducibility and Validation
\end{itemize}

\newthought{Start with an overview} figure that depicts your overall architecture, flow and research design.  
Then proceed to describe the individual components in dedicated sections. The transition between sections need to be motivated and logical.

\newthought{Do not loose touch} with the problem statement and objectives defined in the introduction, make sure to clearly link back to them where appropriate.

\newthought{This chapter} contains the theoretical foundations, models, algorithms, experimental setups, data recorded, metrics, evaluation strategies, or analytical techniques employed in your research.

\marginnote{%
  \newthought{Don't cite!}, reference. 
  The methods section outlines your work, not the work of others. Existing knowledge needs to be described and introduced in the Introduction. 
  If you feel the need to refresh the reader on established methods, do so by referencing back.
}

\section{Results}\index{Results}
\newthought{The results section} is $25 - 35\%$ of the total thesis length and presents the findings of your research.
Results should be presented as objectively as possible, without mixing interpretation or discussion. 
Focus on reporting the findings clearly and concisely, using neutral language and avoiding subjective statements. 

\newthought{Always have several (at least two) benchmarks} to compare your results against. Great benchmarks are human performance, state-of-the-art methods,
established baselines, theoretical limits, or heuristics.

\marginnote{%
  \newthought{Visual aids} such as tables, graphs, and charts are invaluable.
}

\newthought{Structure your results} by grouping related findings and presenting them in a logical order (some might call this story telling) 
that reflects your research objectives. 
Each result should be linked to the corresponding problem statement - again.

\begin{itemize}
  \item Present quantitative results with appropriate statistical measures (e.g., means, standard deviations, confidence intervals).
  \item Use tables to organize numerical data and figures to illustrate trends, patterns, or relationships.
  \item Avoid interpreting or explaining the implications of the results in this section; reserve such commentary for the discussion part.
\end{itemize}

\section{Discussion}\index{Discussion}

\newthought{The discussion section} is $15 - 20\%$ of the total thesis length and interprets and contextualizes your results, linking them back to your research questions and objectives.
A good practice is to make the linkage to your objectives explicit, by again using a bullet point list mirroring the objectives from the introduction.

\marginnote{%
  \newthought{Conclusion}, think: Take-home message.\\
}

\newthought{Feel free to split} the Discussion into a Discussion and Conclusion part. 

\newthought{The discussion} interprets and contextualizes your results and their limitations.
\newthought{While the conclusion} clearly distills the the main contributions of your work, reflects on implications, and points to future work.

\section{Appendices}\index{Appendix}
\newthought{The appendix} contains tables, data, questionnaires, proofs, derivations, and other ancillary information that might 
otherwise negatively affect the flow of the main text. 
The main text must reference the appendix where appropriate. 
Do not use the appendix for outsourcing text that does not fit into the main text due to page restrictions. 
Include an appendix only when necessary.

\makestandardbackmatter{howtothesis_references}

\end{document}
