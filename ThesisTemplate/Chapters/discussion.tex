\newthought{The discussion section} is the place to interpret and contextualize your results, linking them amongst each other. 
Usually things become easier when branching out the \textit{Conclusion} part together with the \textit{Future Research} part. 
While in the Chapter on Results (see Ch.~\ref{ch:Results}) you presented your findings and experiments more or less isolated, and in neutral objectivity, 
this Chapter helps the reader to make meaning out of them. The meaning arises from putting your results in context against benchmarks and baselines. 

\newthought{Short Excursion on Benchmarks and Baselines.}
Baselines, are simpler or previously existing approaches that serve as a starting point for comparison. Such as heuristics, or human performance. 
Benchmarks, on the other hand, are established standards or reference points against which your results can be compared. 
They often represent the best-known performance or widely accepted methods in your field and usually provide means to test against them. The quantitative
calculations are of course done already in the Results Chapter, but the interpretation and discussion happens here.


\marginnote{%
  \newthought{This is the time}, to show-off the deltas you worked so hard for. 
}

\newthought{By comparing your results} against benchmarks and baselines, you can demonstrate and discuss the significance and limitations of your work.
While this is a quantitative exercise at first, the discussion allows for thoughtful qualitative interpretation. 
However, be careful to not over-interpret your results, nor is this the moment to introduce new ideas or hypotheses. 
Another important aspect of the discussion is to openly 
address limitations, white spots, and potential sources of error, meaning expected variances and biases, in your study. 







% ----------------------------------------------------------------------
\section{Conclusion}\label{sec:Conclusion}
\newthought{The conclusion section} serves to succinctly summarize the main contributions of your thesis work.
Clearly distill the main contributions of your work, and reflect on their implications. 
While the discussion section interprets the results in detail, the conclusion is more about showing the reader what they walk away with.


% A contribution ------------------------------------------------------
\newthought{For each objective,} in your Section~\ref{sec:Objectives}, mirror your contributions in dedicated blocks here.
Begin by restating the objective in a single sentence, followed by elaborating on how your methods and results addressed it.


% A contribution ------------------------------------------------------
\newthought{Remember,} this could be a new algorithm, model, framework, dataset, or empirical finding - or something in those lines.


% ----------------------------------------------------------------------
\newthought{Concluding,} make the linkage to your objectives explicit, by using a bullet point list mirroring the objectives from Section~\ref{sec:Objectives}, again.
Start out with a key statement, which summarizes the contribution when taking a holistic view: 
\begin{itemize}
  \item Then, for each objective, briefly discuss how your results contribute to it. Make it one to three sentences per contribution.
  \item If you have artifacts (code, data, models) to share, reference them here and provide permanent accessibility.
\end{itemize}








% ----------------------------------------------------------------------
\section{Recommendations for Future Research}\label{sec:FutureResearch}

\marginnote{%
  \newthought{It is a good habit}, to treat this section like a backlog. Documenting work packages which 
    have been in scope of this thesis, but exceeded the available time or resources. 
    Of course you need to prioritize this backlog before drafting this section.
}

\newthought{Based on the findings and limitations discussed}, outline potential avenues for future research that could build upon your work.
This could include exploring unanswered questions, testing your methods in different contexts, or addressing limitations identified in your study. 
While writing this, keep the next student or researcher in mind who might pick up where you left off 
- knowing everything you know now, what would you recommend them to include in their problem statement (compare with yours in Sec.~\ref{sec:ProblemStatement})?  
