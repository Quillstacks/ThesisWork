\newthought{The results section} introduces experimental setups, experimental design, datasets used, hardware, implementation details, 
and presents the findings of your experiments and analyses around your novel methods - in a quantitative manner. Results should be presented as objectively as possible, without blending in interpretations or discussions at this point. 
Focus on reporting the findings clearly and concisely, using neutral language and avoiding subjective statements.

\begin{table}[!ht]
  \centering
  \caption{Example of a well-formatted table with a clear, concise caption. Make sure to go \textit{full width} for tables and figures, and embrace \textit{horizontal lines}, while avoiding vertical lines.}
  \label{tab:example}
  \begin{tabular*}{\linewidth}{@{\extracolsep{\fill}}lcc}
    \toprule
    \textbf{Category} & \textbf{Value 1} & \textbf{Value 2} \\
    \midrule
    Example A & \textit{42} & 3.14 \\
    Example B & 17 & 2.71 \\
    \bottomrule
  \end{tabular*}
\end{table}

 

\newthought{Always have several (at least two) benchmarks or baselines} to compare your results against. Great baselines are human performance, 
state-of-the-art methods, other established baselines, theoretical limits, or heuristics and simple models for starters. 
Present quantitative results with appropriate statistical measures (e.g., means, standard deviations, confidence intervals).
Deciding what and how to measure is crucial. Aligning your metrics and evaluation criteria with your research objectives is a field of study in its own right.
Whenever possible, use established benchmarks and metrics from the literature (you will have introduced them in your Introduction, in Section~\ref{sec:RelatedWork}) to ensure comparability. 
New developments and designs in this are shows in the Methods Chapter (see Ch.~\ref{ch:Methods}).

\newthought{Use tables to organize} numerical data (see Tab.~\ref{tab:example}) and figures (see Fig.~\ref{fig:linegraph}) to illustrate trends, patterns, or relationships.

% You want to contribute? This is the place you are looking for. 
% I am still trying to find a good library to draw nice figures and plots in LaTeX, or at least include them nicely.
% Bonus points using Tufte Style (maybe that package or library still needs to be written, maybe we should do that).
% Getting this step standardized across theses would be great. Reach out if you think this is something you want to work on.

\begin{figure}[!ht]
  \includegraphics[width=\linewidth]{figures/tufte-line.png}
  \caption{Basic line graph after Tufte. Data emphasized, axes info in title, minimal ticks, reduced non-data ink.}
  \label{fig:linegraph}
\end{figure}


\section{Experimental Design}
\newthought{If you have non-trivial} experimental setups or designs, describe them here. 
This includes the design of experiments used to evaluate your methods.

\section{Implementation Details}
\newthought{This also holds} for details about datasets, hardware, software frameworks, and implementation specifics.
Again, focus on the things that are necessary to understand and reproduce your results - not every package you installed is relevant.
A lot will be documented in your code repository and code documentation anyway, so focus on the important bits here. 
And remember to use margin notes for ancillary information - this is especially powerful in this chapter.

\newthought{A good practice} is to have a code demo ready to showcase your results interactively. This could be a Jupyter notebook, 
a web application, or any other way of hosting software in this century that allows users to engage with your findings and contributions.
Make sure that this demo is properly documented and accessible, ideally set up for easy deployment (better application).
\marginnote{
  \newthought{When working on proprietary} or sensitive data, make use of a toy dataset or environment.
}














\section{First Finding}
\newthought{Structure your results} by grouping related findings and presenting them in a logical order (some might call this story telling) 
that reflects your research objectives. 
\marginnote{
  \newthought{Visual aids} such as images, technical drawings, tables, graphs, and charts are invaluable. 
  Yet, make sure the text itself can transfer the results self-sustained.
}
Each result should be linkable to one or multiple corresponding problem statements.






\section{Second Finding}
\newthought{Every method} presented in the Methods chapter should have corresponding results here.
Often your methods are evaluated along multiple dimensions, such as performance metrics, qualitative observations, and comparative analyses. 
