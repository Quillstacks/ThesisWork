\newthought{Every Chapter}, starts with a block of text, which introduces the chapter and its role in the overall thesis structure. 
At the same time this block of text opens up potential sections within the chapter. As for the introduction this first block takes care 
of the context and motivation, of your thesis. The introduction is $20 - 30\%$ of the total thesis length. 

\marginnote{\newthought{Lengthy motivation sections}\\
  covering how autonomous driving brings down traffic accidents by 90\%, reiteration of other broad concepts, 
  or humble bragging in favor of your industrial partner, or similar topics, are to be avoided - A short single sentence will do.
  Then focus on motivating your specific field and research problem, this is in its own right a broad field.
}

\section{Problem Statement}\label{sec:ProblemStatement}
\newthought{Why is this specific field of research important, now?} Then, quickly narrow down to the specific field you will be addressing.
Broadly outline the current state of the art in this field, meaning all relevant fields for that matter, and how they relate to reach other. Be diligent in preparing the ground for pointing out
gaps you identified in existing knowledge or limitations in current approaches and the overarching problems that arise from the field. 
These gaps are often a combination of several factors, such as limitations of current approaches, gaps in theoretical knowledge, constraints in applications, 
or emerging developments that necessitate further investigation. 

\newthought{Citations} are essential in academic writing to give credit to original sources. 
In your proposal, there is no reference section, instead they will all show up in the marginnotes like so~\cite{SchuteraBachelorThesis}.
This is achieved by use of the \texttt{\textbackslash cite\{\}} command. This is also means that you should limit and focus yourself
to the most relevant works. Later in your thesis you will have a full reference section at the end of your document, and this section will be brimming with citations.

\vfill
\section{Objectives}\label{sec:Objectives}
\marginnote{%
\newthought{Schöpfungshöhe}, a certain level of originality and creativity required in academic work, is closely linked to the nature of your contributions. 
You will soon find yourself trading off between high-risk-high-impact ideas and more conservative incremental objectives. 
I advise to aim for a mix when defining your contributions: A workhorse (driven by execution and rigor), a staircase (small incremental improvement on a known method), 
a moonshot (high-impact idea or novel recombination, which might fail).
} 

\newthought{Open this} with a 3-5 sentences which distill the derived problem statement. 
Then clearly (in 3-5 bullet points) outline the problem statement and articulate the contributions you will be making. 

\newthought{The contributions} should be:

\begin{itemize}
  \item Specific: Clearly define what you aim to contribute.
  \item Measurable: Ensure that your contributions can be evaluated.
  \item Achievable: Set goals that are realistic to be accomplished.
  \item Relevant: Align your contributions with the problem statement.
  \item Time-bound: Specify a timeline for achieving each contribution.
\end{itemize}
\marginnote{
\newthought{Specific contributions}, could sound like: A new method for X, More sensitive metrics for Y, an empirical study on Z, a curated dataset for X, 
a data-driven analysis of Y, or an application of Z for a new domain. After reading the 
bullet points, the reader should be able to clearly understand what you are trying to achieve and what he will be walking away with.
}


\section{Timeline}
\newthought{Present a timeline} that outlines the key milestones and deadlines for your research project.
You can make use of Table~\ref{tab:timeline} as a reference for structuring your own timeline. Make sure to 
update the descriptions to make them actionable for your work. The table does not count into your page limit, 
you will benefit from greater detail here.

\begin{table}[ht]
  \centering
  \begin{tabular}{p{0.23\linewidth} p{0.55\linewidth} p{0.17\linewidth}}
    \toprule
    \textbf{Milestone} & \textbf{Description} & \textbf{Timeline} \\
    \midrule
    Review & Survey recent and seminal works, identify gaps and relevant ressources. & Week 1 \\
    Definition* & Formulate research questions and hypotheses based on literature findings. & Week 2 \\
    Development & Prepare data, develop methods, and conduct experiments. & Week 3--9 \\
    Evaluation & Analyze results, compare with baselines, and interpret findings. & Week 7--11 \\
    Submission & Document methods, results and conclusions; polish for thesis submission. & Week 10--12 \\
    \bottomrule
  \end{tabular}
  \caption{Project timeline and milestones for a 12-week research project (e.g. Bachelor Thesis).}
  \label{tab:timeline}
\end{table}

\marginnote{\newthought{*With this submission} you complete the second milestone.} 

